\documentclass[../main.tex]{subfiles}
\graphicspath{{\subfix{../assets/}}}

\begin{document}
    \section{Utilizarea Procesorului}
    Fișierele \acrshort{vhdl} trebuie încărcate într-un mediu de simulare după care se simulează
    entitatea \texttt{CPU\_TB}. Dacă se aplică modificările necesare proiectul poate fi încărcat
    pe o placă \acrshort{fpga}. \\

    \section{Utilizarea \emph{assembler}-ului}
    \begin{description}
        \item[Nume] \hfill \\
        subleqasm -- \emph{assembler} pentru arhitectura \acrshort{subleq}
        \item[Rezumat] \hfill \\
        subleqasm \emph{infile} [-o outfile] [-s size] [-w wordsize]
        \item[Descriere] \hfill \\
        Traduce un fișier de cod sursă într-un fișier binar executabil pe hardware dedicat cu 
        arhitectură \acrshort{subleq}
        \item[Opțiuni] \hfill
        \begin{itemize}
            \item -o, -{}-output <file> -- numele fișierului binar generat. Dacă nu e precizat atunci fișierul
            se va numi `a.bin'.
            \item -s, -{}-size <number> -- dimensiunea minimă în cuvinte a fișierului binar generat. Dacă
            dimensiunea finală a fișierului e mai mică decăt dimensiunea minimă atunci se scrie valoarea
            0x00 până se atinge valoarea minimă. Dacă nu e precizat atunci dimensiunea minimă e 0.
            \item -w, -{}-word <1--8> -- dimensiunea în octeți a cuvântului. Dacă nu e precizat atunci
            dimensiunea este 8 octeți (64 biți).
        \end{itemize}
    \end{description}

\end{document}