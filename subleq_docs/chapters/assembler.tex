\documentclass[../main.tex]{subfiles}
\graphicspath{{\subfix{../assets/}}}

\begin{document}

\subsection{Considerente generale}
Proiectul a fost dezvoltat în C, mai exact standard ISO C14. Alegerea nu a fost arbitrară, C este un limbaj de
programare foarte puternic și capabil. O alternativă luată în considerare a fost C\# dar stilul de programare
OOP lăsa de dorit și tentația de a complica lucrurile folosind funcționalități de limbaj există.
Tipul de \emph{assempler} ales este \emph{two-pass assembler}. Acest tip este mai ușor de înteles și e mai extensibil decât un
\emph{one-pass asembler}. Asta înseamnă ca execută 2 pași asupra fișierului cu codul sursă. Cei 2 pași au
responsabilități diferite.

Primul pas se ocupă de strângerea de informații necesare pentru pasul 2 și de
transmiterea acestor informații. În mod normal se face printr-un fișier intermediar, în cazul nostru
fișierul intermediar nu este unul standard. Acesta conține o versiune mai simplă a fișierului sursă
în loc să conțină una mai complicată. Etichetele și macro-urile găsite în pasul 1 se păstrează în 
memorie, de unde pasul 2 le accesează. Fișierul intermediar conține o versiune mai curața a fișierului
sursă și macro-urile expandate. Din cauza aceasta opținea de listare a codului final într-un fișier
nu este posibilă, la pasul 2 nu se știe cum arată fișierul sursă. Toate erorile sunt prinse la pasul 1.
Pasul 2 traduce efectiv instrucțiunile în cod mașină, executabil de către procesor.

Pentru a face lucrurile mai simple, o linie poate avea maxim 256 de caractere.
Tot ce se află după este ignorat. Simbolul \texttt{ZERO} este singurul care este scris de \emph{assembler},
este o etichetă către o zonă de memorie ce mereu va avea valoarea 0. Aceasta regulă este consolidată în cod
și o încercare de a modifica valoarea va rezulta într-o eroare de compilare. Este posibil ca din cod să se
facă referire la valoarea lui \acrshort{lc} folosind simbolul `*'. Pentru a face referire la adresa
următoarei instrucțiune se folosește simbolul `?'. Atunci instrucțiunea \texttt{NOP}, o instrucțiune
care nu face nimic, se scrie \texttt{subleq ZERO ZERO ?}.

\subsection{Aspecte importante din cod}
Niciodată nu se alocă memorie în mod dinamic. Informațiile referitoare la etichete și macro-uri sunt salvate
într-un vector alocat static. Referința la o etichetă sau macro se face prin indexul acestora în vector.
\begin{lstlisting}[caption={Referință prin index în structura LINE\_DEF}, language={C}, label={lst:staticptr}]
    typedef int16_t static_ptr;
    ...
    typedef struct line_def {
	    static_ptr label;
	    static_ptr macro;
	    LINE_TYPE type;
	    DIRECTIVE_TYPE directive;
	    PARAM_DEF params[5];
    } LINE_DEF;
    ...
    LABEL_DEF labels[INT16_MAX];
    int labels_count;
    MACRO_DEF macros[INT16_MAX];
    int macros_count;
\end{lstlisting}

În interiorul unui macro ar trebui să fie posibil să definești etichete. Problema este că un macro este
făcut pentru a fi folosit în mai multe locuri. Asta ar duce la definirea multiplă a unei etichete, situație
care rezultă în eroare de compilare. Pentru a rezolva problema, etichetele au acum un domeniu de definiție,
care poate fi global sau macro. O etichetă cu domeniu macro nu are valoare. Când macro-ul este expandat
se crează o etichetă globală cu valoarea curentă a lui \acrshort{lc}, la numele căreia se adaugă un sufix
format din 4 litere și cifre alese la întamplare. În felul acesta o etichetă poate fi definită într-un
macro.
\noindent\begin{minipage}{\linewidth}
\begin{lstlisting}[caption={Tratarea unei etichete în interiorul unui macro}, language={C}, label={lst:padding}]
    int macro_expand(int lc, MACRO_DEF macro, PARAM_DEF* params, FILE* stream) {
        ...
        for (int i = macro.first_line; i < macro.first_line + macro.line_count; i++) {
            LINE_DEF line = lines[i];

            if (line.label != -1) { // line contains a label
			        label = labels[line.label];
			        random_padding_after(label.name, 4, seed);
			        label.value = lc;
			        label.scope = LABEL_SCOPE_GLOBAL;
			        label_add(label);
		        }
        ...
        }
    }
\end{lstlisting}
\end{minipage}

Un macro poate fi folosit în interiorul definiției unui alt macro. Pentru a nu permite crearea unui
ciclu infinit de expandare a fost luată decizia de a nu permite folosirea unui macro înainte de
a fi definit. Pentru a permite folosirea unui macro înainte de a fi definit se poate testa că nu există
cicluri prin parcurgerea grafului orientat al dependințelor dintre macro-uri. Dacă un nod al grafului 
este vizitat de 2 ori atunci există un ciclu și trebuie semnalată o eroare.

\subsection{Sintaxă limbaj}
\subsubsection{Instrucțiune}
Singura instrucțiune este \emph{subleq}, din acest motiv este implicită și se scriu doar parametrii. O instrucțiune
poate fi scrisă cu 3 parametrii, 2 parametrii sau 1 parametru. În cazul cu 3 parametrii se execută instrucțiunea subleq
obișnuită \texttt{subleq A B C}. În cazul cu 2 parametrii, al 3-lea parametru se pune automat `?', \texttt{subleq A B ?}.
În felul acesta tot timpul se trece la instrucțiunea următoare. În cazul cu un singur parametru se întelege
\texttt{subleq A A ?} care e echivalent cu \texttt{A = A - A} sau \texttt{A = 0}. S-a luat în considerare ca
forma cu un parametru să se traducă drept \texttt{ZERO ZERO A}, adică salt necondiționat dar are mai puțină
utilitate decât setarea pe 0. Incrementează \acrshort{lc} cu 3.

\subsubsection{Etichetă}
Orice linie poate avea o etichetă. Trebuie să apară prima pe linie, se delimitează de restul liniei prin
simbolul `:', de exemplu \texttt{init: a a ?}.

\subsubsection{Directive}
Pe baza lucrării \cite{asl} a fost luată decizia ca directivele să înceapă cu caracterul `.'. Directivele prezente în limbaj sunt:
\begin{itemize}
    \item \verb|.ORG <number>|\\
    Are un singur parametru numeric. Setează valoarea lui \acrshort{lc}. Este interpretată în pasul 2 prin
    scrierea de valori 0x00 până se ajunge la noua valoare a lui \acrshort{lc}.
    \item \verb|.DATA <number>/<A>|\\
    Are un singur parametru care poate fi un număr sau o etichetă. Stochează în memorie valoarea precizată și 
    incrementează \acrshort{lc} cu 1. Este interpretată în pasul 2 prin scrierea în memorie a valorii
    numerice sau valorii etichetei.
    \item \verb|.END|\\
    Nu are parametrii. Crează un ciclu infinit. Este interpretată în pasul 1 prin expandarea în instrucțiunea 
    \texttt{ZERO ZERO *}.
    \item \verb|<label> .MACRO <p1> <p2> <p3> <p4> <p5>|\\
    Trebuie să aibă o etichetă ce va acționa drept nume al macro-ului. Poate avea între 0-5 argumente, toate sunt etichete
    Nu incrementează \acrshort{lc} și nici liniile următoare nu îl incrementează. După această directivă pot să
    urmeze oricâte linii care conțin instrucțiuni, alte directive sau macro-uri. Aceste linii formează corpul
    macro-ului. Este interpretată în pasul 1 prin setarea \allowbreak \texttt{inside\_macro\_definition = true;}
    \item \verb|.ENDM|\\
    Nu are parametrii. Marchează finalul definiției unui macro. Nu incrementează \acrshort{lc} dar liniile ce urmează
    îl incrementează. Este interpretată în pasul 1 prin setarea \allowbreak \texttt{inside\_macro\_definition = false;}
\end{itemize}

\subsubsection{MACRO}
Pentru a folosi un macro, numele lui trebuie precedat de simbolul `@'. Îl face mai vizibil pentru programator 
și totodată mai ușor de identificat în cod. Un macro poate fi folosit inclusiv în interiorul definiției unui alt
macro dar nu în propria lui definiție.
\begin{lstlisting}[caption={Exemplu macro}, label={lst:macro}]
    NOP: .MACRO
        ZERO
    .ENDM

    ADD: .MACRO a b
        b r1
        r1 a
        r1
    .ENDM

    @ADD x y
    @NOP
    @NOP

    x:  .DATA 5
    y:  .DATA 3
    r1: .DATA 0
\end{lstlisting}

\subsubsection{Comentarii}
Suportă doar simboluri de o linie. Caracterul `;' e desemnat pentru a marca începutul unui comentariu. Totodată,
din cauză că o linie are doar 256 de caractere, tot ce se află dupa acele 256 de caractere e ignorat. Textul care urmează
după o instrucțiune cu 3 parametrii sau după ultimul argument al unei directive este de asemenea ignorat. Se recomandă
ca toate comentariile să înceapă cu `;' și să nu se depășească niciodată 256 de caractere pe o linie, deși nu se
generează o eroare.

\noindent\begin{minipage}{\linewidth}
\begin{lstlisting}[caption={Exemplu de comentarii}, label={lst:comment}]
    A ;this is a comment
    A B                                 consider this to be after 256 characters
    A B C this is also a comment
    m1: .DATA -1 this is also a comment
\end{lstlisting}
\end{minipage}

\subsection{Exemplu cod}

\begin{lstlisting}[caption={Exemplu de cod în subleq assembly}, label={lst:code}]
    reset: .MACRO p
        p
    .ENDM
    
    sub: .MACRO a b
        b a
    .ENDM
    
    add: .MACRO a b
        @SUB r1 b
        @SUB a r1
        @reset r1
    .ENDM
    
    MULT: .MACRO a b
        @SUB r1 b
        @SUB r1 one
        @SUB r2 a
        @RESET a
        loop: @SUB a r2
        one r1 loop
        @RESET r1
        @RESET r2
    .ENDM
    
    ;// FIRST CORE PROGRAM //
    @SUB b a
    @RESET a
    .END
    
    a: .DATA 4
    b: .DATA 2
    
    ;-------------------------------------------------------------------
    
    ;// SECOND CORE PROGRAM //
    .ORG 128
    @MULT x y
    @MULT x y
    .END
    
    x:  .DATA 3
    y:  .DATA 2
    r1: .DATA 0
    r2: .DATA 0
    ONE: .DATA -1
    
    ; at the end will be added symbol ZERO which is common for the cores
\end{lstlisting}

\begin{lstlisting}[caption={Conținutul fișierului intermediar pentru exemplul dat}, label={lst:intermediate}]
    a b ? 
    a a ? 
    ZERO ZERO *
    .DATA 4
    .DATA 2
    .ORG 128
    y r1 ? 
    one r1 ? 
    x r2 ? 
    x x ? 
    r2 x ? 
    one r1 loopWJMS 
    r1 r1 ? 
    r2 r2 ? 
    y r1 ? 
    one r1 ? 
    x r2 ? 
    x x ? 
    r2 x ? 
    one r1 loop4VZY 
    r1 r1 ? 
    r2 r2 ? 
    ZERO ZERO *
    .DATA 3
    .DATA 2
    .DATA 0
    .DATA 0
    .DATA -1
    .DATA 0
\end{lstlisting}

\subsection{Raportarea erorilor}
În limbaj sunt documentate 12 tipuri de erori care pot să apară Acestea sunt:
\begin{itemize}
    \item \verb|ERROR_INVALID_DATA_PARAM| -- parametru invalid pentru directiva \verb|.DATA|, acceptă doar numere și nume valide de etichete
    \item \verb|ERROR_INVALID_SYMBOL_NAME| -- nume invalid de etichetă, trebuie să înceapă cu o literă și poate conține cifre, nu poate conține simboluri speciale, maxim 16 caractere
    \item \verb|ERROR_UNKNOWN_DIRECTIVE| -- directivă necunoscută, nu se află în lista de directive
    \item \verb|ERROR_MULTIPLY_DEFINED_LABEL| -- o etichetă nu are voie să fie definită de mai multe ori
    \item \verb|ERROR_UNDEFINED_SYMBOL| -- o etichetă sau un macro e folosit fără a fi definit undeva în cod
    \item \verb|ERROR_INTERNAL_SYMBOL_REDEFINED| -- \verb|ZERO| e singurul simbol de care se ocupă \emph{assembler}-ul, e o etichetă către o zonă de memorie ce mereu conține valoarea 0x00, nu poate fi redefinită
    \item \verb|ERROR_SYMBOL_ZERO_READONLY| -- la adresa simbolului \verb|ZERO| este mereu valoarea 0, nu este permisă schimbarea valorii
    \item \verb|ERROR_MACRO_INSIDE_MACRO| -- un macro nu poate fi definit în interiorul unui macro
    \item \verb|ERROR_ENDM_OUTSIDE_MACRO| -- \verb|ENDM| trebuie să fie precedat de un \verb|.MACRO|
    \item \verb|ERROR_MACRO_NAME_MISSING| -- o linie ce definește un macro trebuie să aibă o etichetă, aceasta devine numele macro-ului
    \item \verb|ERROR_TOO_FEW_ARGUMENTS| -- la folosirea unui macro au fost furnizate prea putine argumente
    \item \verb|ERROR_TOO_MANY_ARGUMENTS| -- la folosirea unui macro au fost furnizate prea multe argumente
\end{itemize}
    
\end{document}