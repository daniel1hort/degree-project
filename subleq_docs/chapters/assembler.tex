\documentclass[../main.tex]{subfiles}
\graphicspath{{\subfix{../assets/}}}

\begin{document}

\subsection{Considerente generale}
Proiectul a fost dezvoltat în C, mai exact standard ISO C14.
Despre .ZERO
maxim 256 caractere pe linie
cei 2 pasi

\subsection{Aspecte importante din cod}
fara memorie alocata dinamic
simboluri speciale
macro nerecursiv
label in macro

\subsection{Sintaxă limbaj}
\subsubsection{Instrucțiune}
Singura instrucțiune este \emph{subleq}, din acest motiv este implicită și se scriu doar parametrii. O instrucțiune
poate fi scrisă cu 3 parametrii, 2 paametrii sau 1 parametru. În cazul cu 3 parametrii se execută instrucțiunea subleq
obișnuită \texttt{subleq A B C}. În cazul cu 2 parametrii, al 3-lea parametru se pune automat `?', \texttt{subleq A B ?}.
nstrucțiunea următoare. În cazul cu un singur parametru se întelegeÎn felul acesta tot timpul se trece la i
\texttt{subleq A A ?} care e echivalent cu \texttt{A = A - A} sau \texttt{A = 0}. S-a luat în considerare ca
forma cu un parametru să se traducă drept \texttt{ZERO ZERO A}, adică salt necondiționat dar are mai puțină
utilitate decât setarea pe 0. Incrementează \acrshort{lc} cu 3.

\subsubsection{Directive}
Pe baza \cite{asl} a fost luată decizia ca directivele să înceapă cu caracterul `.'. Directivele prezente în limbaj sunt:
\begin{itemize}
    \item \verb|.ORG <number>|\\
    Are un singur parametru numeric. Setează valoarea lui \acrshort{lc}.
    \item \verb|.DATA <number>/<A>|\\
    Are un singur parametru care poate fi un număr sau o etichetă. Stochează în memorie valoarea precizată și 
    Incrementează \acrshort{lc} cu 1.
    \item \verb|.END|\\
    Nu are parametrii. Crează un ciclu infinit. Se traduce într-o instrucțiune, \texttt{ZERO ZERO *}.
    \item \verb|<label> .MACRO <p1> <p2> <p3> <p4> <p5>|\\
    Trebuie să aibă o etichetă ce va acționa drept nume al macro-ului. Poate avea între 0-5 argumente, toate sunt adrese
    Nu incrementează \acrshort{lc} și nici liniile următoare nu îl incrementează. După această directivă pot să
    urmeze oricâte linii care conțin instrucțiuni, alte directive sau macro-uri. Aceste linii formează corpul
    macro-ului.
    \item \verb|.ENDM|\\
    Nu are parametrii. Marchează finalul definiției unui macro. Nu incrementează \acrshort{lc} dar liniile ce urmează
    îl incrementează
\end{itemize}

\subsubsection{MACRO}

\subsubsection{Comentarii}
Suportă doar simboluri de o linie. Caracterul `;' e desemnat pentru a marca începutul unui comentariu. Totodată,
din cauză că o linie are doar 256 de caractere, tot ce se află dupa acele 256 de caractere e ignorat. Textul care urmează
după o instrucțiune cu 3 parametrii sau după ultimul argument al unei directive este de asemenea ignorat. Se recomandă
ca toate comentariile să înceapă cu `;' și să nu se depășească niciodată 256 de caractere pe o linie, deși nu se
generează o eroare.
\begin{lstlisting}[caption={Exemplu de comentarii}, label={lst:comment}]
    A ;this is a comment
    A B                                 consider this to be after 256 characters
    A B C this is also a comment
    m1: .DATA -1 this is also a comment
\end{lstlisting}

\begin{lstlisting}[caption={Exemplu de cod în subleq assembly}, label={lst:code}]
    reset: .MACRO p
        p
    .ENDM
    
    sub: .MACRO a b
        b a
    .ENDM
    
    add: .MACRO a b
        @SUB r1 b
        @SUB a r1
        @reset r1
    .ENDM
    
    MULT: .MACRO a b
        @SUB r1 b
        @SUB r1 one
        @SUB r2 a
        @RESET a
        loop: @SUB a r2
        one r1 loop
        @RESET r1
        @RESET r2
    .ENDM
    
    ;// FIRST CORE PROGRAM //
    @SUB b a
    @RESET a
    .END
    
    a: .DATA 4
    b: .DATA 2
    
    ;-------------------------------------------------------------------
    
    ;// SECOND CORE PROGRAM //
    .ORG 128
    @MULT x y
    @MULT x y
    .END
    
    x:  .DATA 3
    y:  .DATA 2
    r1: .DATA 0
    r2: .DATA 0
    ONE: .DATA -1
    
    ; at the end will be added symbol ZERO which is common for the cores
    \end{lstlisting}

\subsection{Raportarea erorilor}
În limbaj sunt documentate 12 tipuri de erori care pot să apară Acestea sunt:
\begin{itemize}
    \item \verb|ERROR_INVALID_DATA_PARAM| -- parametru invalid pentru directiva \verb|.DATA|, acceptă doar numere și nume valide de etichete
    \item \verb|ERROR_INVALID_SYMBOL_NAME| -- nume invalid de etichetă, trebuie să înceapă cu o literă și poate conține cifre, nu poate conține simboluri speciale, maxim 16 caractere
    \item \verb|ERROR_UNKNOWN_DIRECTIVE| -- directivă necunoscută, nu se află în lista de directive
    \item \verb|ERROR_MULTIPLY_DEFINED_LABEL| -- o etichetă nu are voie să fie definită de mai multe ori
    \item \verb|ERROR_UNDEFINED_SYMBOL| -- o etichetă sau un macro e folosit fără a fi definit undeva în cod
    \item \verb|ERROR_INTERNAL_SYMBOL_REDEFINED| -- \verb|ZERO| e singurul simbol de care se ocupă \emph{assembler}-ul, e o etichetă către o zonă de memorie ce mereu conține valoarea 0x00, nu poate fi redefinită
    \item \verb|ERROR_SYMBOL_ZERO_READONLY| -- la adresa simbolului \verb|ZERO| este mereu valoarea 0, nu este permisă schimbarea valorii
    \item \verb|ERROR_MACRO_INSIDE_MACRO| -- un macro nu poate fi definit în interiorul unui macro
    \item \verb|ERROR_ENDM_OUTSIDE_MACRO| -- \verb|ENDM| trebuie să fie precedat de un \verb|.MACRO|
    \item \verb|ERROR_MACRO_NAME_MISSING| -- o linie ce definește un macro trebuie să aibă o etichetă, aceasta devine numele macro-ului
    \item \verb|ERROR_TOO_FEW_ARGUMENTS| -- la folosirea unui macro au fost furnizate prea putine argumente
    \item \verb|ERROR_TOO_MANY_ARGUMENTS| -- la folosirea unui macro au fost furnizate prea multe argumente
\end{itemize}
    
\end{document}