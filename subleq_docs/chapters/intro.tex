\documentclass[../main.tex]{subfiles}
\graphicspath{{\subfix{../assets/}}}

\begin{document}

Aparatele de calcul au prins o perioadă rapidă de evoluție la începutul anilor 1940 odată cu apariția primului
calculator care a folosit tuburi electronice cu vid. 10 ani mai târziu avem testul Turing, creat de matematicianul
britanic Alan Turing și primul procesor comercial, apoi primul procesor care folosește tranzistori și primul
limbaj de programare de nivel înalt, \acrfull{fortran}, dezvoltat la IBM. Totul s-a schimbat în anii 1970, când
microprocesoarele intră în scenă. Până atunci toate procesoarele erau pentru uz comercial sau intern într-o companie,
erau scumpe, aveau un consum ridicat dar erau foarte puternice pentru standardele de atunci. Microprocesoarele au
revoluționat industria deschizând piața pentru aparate electronice de uz casnic și individual. Având mai puține
instrucțiuni dar un cost mai mic de producție, erau perfecte și fezabile pentru a le integra în autovehicule,
aparate casnice, console portabile de jocuri, etc. În prezent putem purta acceași discuție despre \acrshort{cisc}
vs \acrshort{risc}.

Limbajele de programare au evoluat și ele în timp ca să fie cât mai ușor de folosit. Dar există o categorie care
s-a axat pe a face munca programatorului mai grea sau limbaje create pentru a experimenta un fel nou de programare.
\emph{``An esoteric programming language (ess-oh-terr-ick), or esolang, is a computer programming language designed to 
experiment with weird ideas, to be hard to program in, or as a joke, rather than for practical use.''}\cite{esolangs}.
Aceste limbaje au ca scop cel puțin una dintre următoarele:
\textit{
\begin{itemize}
    \item Minimalism
    \item New concepts
    \item Weirdness
    \item Themed
    \item Brevity
    \item Jokes
    \item Obfuscation
\end{itemize}}

La capitolul Minimalism putem vorbi de o întreagă specie de limbaje ezoterice, \acrfull{oisc} \cite{oisc}. Dar pe noi
ne interesează \textbf{\acrfull{subleq}} \cite{subleq}, din simplul motiv că e cel mai popular. Cum funcționează
\acrshort{subleq} și cât de popular este? Subleq e singura instructiune din limbaj, care poate fi considerat un limbaj
de asamblare. Are 3 parametrii $A$ $B$ și $C$, toți sunt adrese. Procesorul va executa operația:
\begin{equation*}
    MEM[B] = MEM[B] - MEM[A]
\end{equation*}
și dacă rezultatul operației este mai mic sau egal cu 0 atunci procesorul sare cu execuția programului la adresa $C$.

\acrshort{subleq} e atât de popular încât oamenii au început să scrie lucrări despre el, sa construiască sau să simuleze procesoare
pe aceasta arhitectură, de asemenea și această lucrare e la fel. Există asambloare, există chiar mai multe versiuni ale
limbajului, compilatoare și chiar un sistem de operare, DawnOS \cite{dawn}. Următoarele paragrafele conțin câteva dintre 
motivațiile creatorului DawnOS \cite{dawn}, Geri. Sunt răspunsuri la întrebarea:
``\emph{Why technologies like Dawn and SUBLEQ are important?}''.

\emph{``Your desktop computer have more than one billion of transistors in it, and it consumes 100 watts to read this text on it.}'',

\emph{``Because the detailed instruction manual of todays desktop (x86) CPU-s and IO system is approximately 50 000 page long, and no one completely understands how they work.}''

\emph{``Computers are driven by software written from 400 million lines of source code.''}

În capitolele ce urmează vom explora mai multe lucruri interesante despre \acrshort{oisc} \cite{oisc}, \textbf{implementarea și
simularea unui procesor multi-core cu arhitectura \acrshort{subleq} \cite{subleq}}, implementarea unui asamblor pentru a putea
scrie cod ca să testăm procesorul, sintaxa limbajului și modul de utilizare a asamblorului, urmate de concluziile proiectului.

\end{document}