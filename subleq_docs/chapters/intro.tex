\documentclass[../main.tex]{subfiles}
\graphicspath{{\subfix{../assets/}}}

\begin{document}

Sunt șanse mari să citești această lucrare pe un ecran. Dacă mă înșel te rog să mă scuzi, 
de fapt, poți chiar să sari la următorul paragraf, deoarece nu va face sens ce urmează să spun
dacă ai ales să listezi documentul de față. Dacă mai ești aici înseamnă că, probabil, folosești un dispozitiv,
numit în limbaj uzual drept calculator, pentru a citi aceste cuvinte. La baza acestui calculator
se află procesorul sau \acrshort{cpu}. Nu din întâmplare procesorul se află și la baza acestei
lucrări. 

În continuare vom explora cum funcționează un procesor, ce tipuri de arhitecturi există,
vom descoperi că o instrucțiune e tot ce ai nevoie, chiar și mai puțin de atât și vom scrie
niste programe într-un limbaj de asamblare creat special pentru procesorul pe care l-am făcut.

Pentru acest proiect am decis să proiectez un procesor \emph{multi-core} cu arhitectură
\acrshort{subleq}\cite{subleq} în \acrshort{vhdl}. Pentru a testa procesorul voi realiza și un \emph{assembler}
care să ia codul scris în limbaj de asamblare și să îl traducă în instrucțiuni pe care
acesta le poate executa.

\acrshort{subleq}\cite{subleq} vine de la \emph{\acrlong{subleq}} și este o arhitectură/limbaj ce face parte
din categoria \acrfull{oisc}. Singura instructiune din arhitectura \acrshort{subleq} este subleq,
care facilitează scăderea ca operatie aritmetică pentru prelucrarea datelor dar și salt condiționat
de rezultatul scăderii. Vom dezvolta mai mult în capitolele următoare.

``\emph{Your desktop computer have more than one billion of transistors in it, and it consumes 100 watts to read this text on it.}'',
``\emph{Because the detailed instruction manual of todays desktop (x86) CPU-s and IO system is approximately 50 000 page long, and no one completely understands how they work.}''
\cite{dawn}. Motivațiile pentru existența unei astfel de arhitecturi sunt diverse. Pentru unii e simplă curiozitate,
aceștia se întreabă dacă e posibil să existe o arhitectură de tip \acrshort{oisc}. Pentru alții acționează ca
instrument didactic, până la urmă e un concept foarte simplu și ușor de analizat și de predat. Pentru Geri,
creatorul sistemului de operare \emph{DawnOS}\cite{dawn}, e o alternativă cu un cost de producție foarte redus,
consum minim de resurse și foarte ușor de înțeles. Cele două fraze de la începutul paragrafului sunt doar câteva
din răspunsurile lui Geri la întrebarea: ``\emph{Why technologies like Dawn and SUBLEQ are important?}''

\end{document}