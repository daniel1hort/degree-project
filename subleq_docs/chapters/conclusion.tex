\documentclass[../main.tex]{subfiles}
\graphicspath{{\subfix{../assets/}}}

\begin{document}
    În acest proiect am realizat un procesor cu 2 core-uri cu arhitectura \acrshort{subleq} și un
    \emph{assembler} pentru a scrie cod cu care să fie testat procesorul. Cele 2 core-uri execută
    programe diferite pentru a face circuitul mai simplu. Am ales să fac ceva cu totul nou pentru
    mine și să lucrez cu o tehnologie cu care nu sunt familiar. Procesorul a rezultat în a fi
    mult simplificat față de idea originală pe care am avut-o din cauza lipsei mele de experiență
    în \acrshort{vhdl}. în același timp \emph{assembler}-ul este mai complex decât era inițial în plan
    să fie. Implementarea memoriei cache și a macro-urilor au fost cele mai interesante dar și
    dificile sarcini ale proiectului. De fapt scrierea documentației a fost cea mai grea.

    Pentru îmbunătățirea proiectului s-a luat în considerare schimbarea mapării memoriei
    cache în mapare asociativă pe seturi și implementarea politicii de înlocuire \acrshort{lru}.
    De asemenea sistemul de operare DawnOS prezintă o listă de specificații tehnice care
    dacă sunt implementate procesorul va fi în stare să ruleze sistemul de operare. Circuitul nu a 
    fost încărcat pe o placă \acrshort{fpga} din păcate dar acest lucru ar trebui să fie relativ simplu
    de realizat.

    \emph{Assembler}-ul conține multe funcționalități menite să ajute programatorul în scrierea
    programelor atâta timp cât totul e scris într-un singur fișier. Pentru viitor pot fi
    implementate funcții, ceea ce ar permite crearea unui \emph{entry-point}, precum e
    funcția \emph{main} în C/C++. De asemenea un \emph{loader} va permite compilarea mai multor
    fișiere sursă înt-un singur fișier binar executabil. Astfel proiecte complexe si biblioteci
    refolosibile pot fi dezvoltate. Ulterior poate fi implementat și un limbaj de nivel
    înalt.

    În concluzie acest proiect este unul reușit, deși nu am făcut tot ce mi-am propus.
    Sunt conștient de greșelile făcute în timpul dezvoltării proiectului și sunt dispus să le rezolv.
    Acestea fiind spuse, voi continua lucrul la acest proiect deoarece a fost o experiență foarte
    distractivă și educativă dar și pentru că este ceva ce nu mi-aș fi putut imagina că voi
    realiza când am început studiile de licență.
\end{document}