\documentclass[12pt, a4paper]{uptacthesis}
\usepackage[romanian]{babel}
\graphicspath{{assets/}}
\usepackage{subfiles}

\usepackage[toc, acronym]{glossaries}
\makenoidxglossaries
\newacronym{subleq}{SUBLEQ}{SUBstract and branch if Less than or EQual to 0}
\newacronym{cisc}{CISC}{Complex Instruction Set Computer}
\newacronym{risc}{RISC}{Reduced Instruction Set Computer}
\newacronym{cpu}{CPU}{Central Processing Unit}
\newacronym{vhdl}{VHDL}{Very High Speed Integrated Circuit Hardware Description Language}
\newacronym{oisc}{OISC}{One Instruction Set Computer}
\newacronym{urisc}{URISC}{Ultimate Reduced Instruction Set Computer}
\newacronym{p1eq}{P1EQ}{Plus 1 and brach if EQal}
\newacronym{addleq}{ADDLEQ}{ADD and branch if Less than or EQal to 0}
\newacronym{fpga}{FPGA}{Field-Programmable Gate Array}
\newacronym{spi}{SPI}{Serial Peripheral Interface}
\newacronym{vi}{VI}{Valid-Invalid}

\title{Proiectarea și simularea unui procesor multi-core cu arhitectură SUBLEQ}
\author{\large\bfseries
    Candidat: Daniel-Bogdan Horț\\[1ex]
    Coordonator științific: Ș.l.dr.ing.~Eugen Gurban\\[2cm]
}
\date{\large Sesiunea 2022}

\begin{document}
\maketitle

\chapter*{Rezumat}
\subfile{chapters/abstract}

\tableofcontents

\chapter{Introducere}
\subfile{chapters/intro}

\chapter{Studiu bibliografic}
\subfile{chapters/others}

\chapter{Fundamentare teoretica}
\subfile{chapters/theory}

\chapter{Implementare}
\section{Procesor}
\subfile{chapters/cpu}
\section{Assembler}
\subfile{chapters/assembler}

\chapter{Manual de utilizare}
\subfile{chapters/manual}

\chapter{Concluzii}
\subfile{chapters/conclusion}

\bibliographystyle{plain}
\bibliography{refs}

\cleardoublepage
\addcontentsline{toc}{chapter}{\listfigurename}
\listoffigures

\printnoidxglossary[type=\acronymtype, title=Listă de abrevieri]

\appendix
\cleardoublepage
\subfile{chapters/addendum}
\end{document}